\chapter{Sistema Propuesto}\label{capit:cap3}
\vspace{-2.0325ex}%
\noindent
\rule{\textwidth}{0.5pt}
\vspace{-5.5ex}% 
\newcommand{\pushline}{\Indp}% Indent puede ir o no :p

\section{Adquisición de los datos}
Lorem ipsum dolor sit amet, consectetur adipiscing elit. Aliquam sit amet lobortis turpis. Praesent auctor mi metus, sed bibendum ligula efficitur eu. Suspendisse ut ante id erat interdum accumsan. Pellentesque eget hendrerit eros, et ullamcorper elit. Proin a lacus et sem hendrerit efficitur. Praesent eget eros sed tellus dapibus bibendum sit amet vel justo. Maecenas finibus porttitor dictum. Fusce lacinia dictum interdum.

\begin{center}[a]
\begin{tabular}{ c c c }
 cell1 & cell2 & cell3 \\ 
 cell4 & cell5 & cell6 \\  
 cell7 & cell8 & cell9    
\end{tabular}
\end{center}

\begin{center}[b]
\begin{tabular}{ |c|c|c| } 
 \hline
 cell1 & cell2 & cell3 \\ 
 cell4 & cell5 & cell6 \\ 
 cell7 & cell8 & cell9 \\ 
 \hline
\end{tabular}
\end{center}

\begin{center}[c]
 \begin{tabular}{||c c c c||} 
 \hline
 Col1 & Col2 & Col2 & Col3 \\ [0.5ex] 
 \hline\hline
 1 & 6 & 87837 & 787 \\ 
 \hline
 2 & 7 & 78 & 5415 \\
 \hline
 3 & 545 & 778 & 7507 \\
 \hline
 4 & 545 & 18744 & 7560 \\
 \hline
 5 & 88 & 788 & 6344 \\ [1ex] 
 \hline
\end{tabular}
\end{center}

\begin{center}[d]
\begin{tabular}{ | b{5cm} | m{2cm}| m{2cm} | } 
\hline
The aligning options are m for middle, p for top and b for bottom.& cell2 & cell3 \\ 
\hline
cell1 dummy text dummy text dummy text & cell5 & cell6 \\ 
\hline
cell7 & cell8 & cell9 \\ 
\hline
\end{tabular}
\end{center}

\begin{center}[e]
\begin{tabular}{ |p{3cm}||p{3cm}|p{3cm}|p{3cm}|  }
 \hline
 \multicolumn{4}{|c|}{Country List} \\
 \hline
 Country Name     or Area Name& ISO ALPHA 2 Code &ISO ALPHA 3 Code&ISO numeric Code\\
 \hline
 Afghanistan   & AF    &AFG&   004\\
 Aland Islands&   AX  & ALA   &248\\
 Albania &AL & ALB&  008\\
 Algeria    &DZ & DZA&  012\\
 American Samoa&   AS  & ASM&016\\
 Andorra& AD  & AND   &020\\
 Angola& AO  & AGO&024\\
 \hline
\end{tabular}
\end{center}


\begin{center}[f]
\begin{tabular}{ |c|c|c|c| } 
\hline
col1 & col2 & col3 \\
\hline
\multirow{3}{4em}{Multiple row} & cell2 & cell3 \\ 
& cell5 & cell6 \\ 
& cell8 & cell9 \\ 
\hline
\end{tabular}
\end{center}

\section{Detección}
Nunc hendrerit justo vitae leo imperdiet, eu egestas nunc tristique. Etiam eget risus purus. Suspendisse sagittis tellus eu ipsum ultrices porttitor. Aliquam iaculis, metus sed ullamcorper blandit, justo nibh vehicula ipsum, vitae finibus diam orci vitae magna. Donec sit amet orci a dui laoreet euismod. Sed sed justo eget metus fermentum lacinia quis eget tellus. Pellentesque nibh metus, auctor id felis sed, lobortis condimentum urna. Nullam vel pharetra nisi. Sed volutpat nisi at efficitur blandit. Nulla interdum dictum dui, nec laoreet diam vulputate non. Lorem ipsum dolor sit amet, consectetur adipiscing elit. Suspendisse non lobortis elit, vel bibendum tellus. Praesent gravida feugiat metus, non ultricies nunc mattis ut \ref{tab:recopilacionDeCuestionarios}


\begin{table}
\footnotesize
\centering
\caption{Mauris et imperdiet     tortor. Maecenas consectetur lacus elit, dignissim eleifend dolor  ornare ut. Aenean euismod porta nisi, et volutpat ex laoreet sit amet. Sed ac elit vestibulum neque ultrices feugiat}
\label{tab:recopilacionDeCuestionarios}
%\rotatebox{90}{
\begin{tabular}{m{0.2cm}m{2.5cm}m{2.5cm}m{2.5cm}m{2.5cm}m{2.5cm}}
\hline\noalign{\smallskip}
 & \textbf{FFS} & \textbf{SOFA} & \textbf{FQ} & \textbf{CIS20R} & \textbf{FACIT}
\\ \noalign{\smallskip}
\hline
\noalign{\smallskip}
1	&	TAF	&	TAF	&	PF	&	PF	&	PF\\
2	&	TAF	&	CM	&	CS	&	EE	&	PF\\
3	&	PF	&	TAF	&	CS	&	CM	&	EE\\
\hline
\end{tabular}
%}
\end{table}



\section{Extracci\'on de caracter\'isticas}\label{secc:pruebasFisicas}
Lorem ipsum dolor sit amet, consectetur adipiscing elit. Aliquam sit amet lobortis turpis. Praesent auctor mi metus, sed bibendum ligula efficitur eu. Suspendisse ut ante id erat interdum accumsan. Pellentesque eget hendrerit eros, et ullamcorper elit. Proin a lacus et sem hendrerit efficitur. Praesent eget eros sed tellus dapibus bibendum sit amet vel justo. Maecenas finibus porttitor dictum. Fusce lacinia dictum interdum (Ecuación \ref{eq:demandaOxigeno_simple}).


\begin{equation} \label{eq:demandaOxigeno_simple}
\begin{split} 
& K = R + H + V \\ 
& R = \textrm{consumo de oxígeno} \times kg^{-1} \times min^{-1}\\ 
& H = \textrm{constante horizontal} \times \textrm{velocidad de desplazamiento}\\ 
& V = \textrm{constante vertical} \times \textrm{velocidad de desplazamiento}\\ 
\end{split} 
\end{equation} 

Lorem ipsum dolor sit amet, consectetur adipiscing elit. Aliquam sit amet lobortis turpis. Praesent auctor mi metus, sed bibendum ligula efficitur eu. Suspendisse ut ante id erat interdum accumsan. Lorem ipsum dolor sit amet, consectetur adipiscing elit. Aliquam sit amet lobortis turpis. Praesent auctor mi metus, sed bibendum ligula efficitur eu. Suspendisse ut ante id erat interdum accumsan. Lorem ipsum dolor sit amet, consectetur adipiscing elit. Aliquam sit amet lobortis turpis. Praesent auctor mi metus, sed bibendum ligula efficitur eu. Suspendisse ut ante id erat interdum accumsan. Pellentesque eget hendrerit eros, et ullamcorper elit. Proin a lacus et sem hendrerit efficitur. Praesent eget eros sed tellus dapibus bibendum sit amet vel justo. Maecenas finibus porttitor dictum. Fusce lacinia dictum interdum (Ecuación \ref{eq:testing}).

\begin{figure}\label{eq:testing}
\[ e = m c^2 \]
\caption{A $ \dfrac{3}{2} $ famous equation, where $ e = m c^2  $ represent $ e = m c^2  $ energy something and $  A_{d} = -g - \frac{\sum F}{mass} $  another $\protect\pi=\protect\varpi + \protect\xi$ thing. }
\end{figure}


\section{Reconocimiento}\label{secc:disenoMetodoFatiga}

Lorem ipsum dolor sit amet, consectetur adipiscing elit. Aliquam sit amet lobortis turpis. Praesent auctor mi metus, sed bibendum ligula efficitur eu. Suspendisse ut ante id erat interdum accumsan. Pellentesque eget hendrerit eros, et ullamcorper elit. Proin a lacus et sem hendrerit efficitur. Praesent eget eros sed tellus dapibus bibendum sit amet vel justo. Maecenas finibus porttitor dictum. Fusce lacinia dictum interdum.

\subsection{Extracción de parámetros}
Ecuación \ref{eq:gravedad}. 

\begin{equation} \label{eq:gravedad}
A_{d} = -g - \frac{\sum F}{mass}
\end{equation} 


 


Donde $A_{d}$ representan la aceleración que se aplica a un dispositivo, $g$ la constante de gravedad de 9.81 m/$s^{2}$, y $\sum F$ las fuerzas que se aplican al propia sensor.

\subsection{Cálculo de la demanda de oxígeno} \label{secc:calculoDemandaOxigeno}
Text  \ref{eq:demandaOxigeno_personalizada} and more $O_{2}$ text:

\begin{equation} \label{eq:demandaOxigeno_personalizada}
\begin{split} 
& K = R + H + V \\ 
& R = 3.5 - (0.0367 \times BMI) - (0.0038 \times age) + (0.1790 \times gender)\\
& H = 0.1 \times \textrm{velocidad de desplazamiento}\\ 
& V = 1.8 \times \textrm{velocidad de desplazamiento}\\ 
\end{split} 
\end{equation} 

Donde $1 + 2$ representan el consumo de $O_{2}$ en reposo personalizado al usuario $(ml \times kg^{-1} \times min^{-1})$ \citep{Barstow:1991}, $H$ el componente horizontal relativo a la velocidad de desplazamiento (m/min), $V$ el componente vertical relativo a la velocidad (m/min) y pendiente de desplazamiento (\%).

\begin{description}
  \item[Velocidad:] \hfill \\
  	Para obtener la velocidad de desplazamiento se utiliza el número de pasos realizados por el usuario como se muestra a continuación (Ecuación \ref{eq:velocidad}).
  
\begin{equation} \label{eq:velocidad}
  S_{k} = D_{k}/W \hspace{10 mm} 
  D_{k} = ST_{k} \times SL \hspace{10 mm}
  SL = D_{total}/ST_{total} 
\end{equation}

\end{description}
	
\newpage
%%=====================================================

