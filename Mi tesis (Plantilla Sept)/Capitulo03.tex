\chapter{Sistema Propuesto}\label{capit:cap3}
\vspace{-2.0325ex}%
\noindent
\rule{\textwidth}{0.5pt}
\vspace{-5.5ex}% 
\newcommand{\pushline}{\Indp}% Indent puede ir o no :p

El sistema de reconocimiento de gestos que se propone consta de cuatro etapas: las adquisición de los datos, la detección, extracción de características y el reconocimiento. 

Enseguida se describirán cada una de las etapas. 

\section{Adquisición de los datos}

La captura de los datos se realiza utilizando dos dispositivos Kinect, se toman los datos del sensor de profundidad, estos son representados como una imagen en escala de grises de 8 bits.  

Cada cuadro capturado por el dispositivo es representado de la manera anterior, en las imágenes se puede apreciar pequeños cambios detectados al sensor gracias a la forma en que la imagen es representada. Para lograr esta representación el rango de escala de grises inicia cada 26 cm. es por eso que se pueden apreciar cambios pequeños en las distancias entre los objetos. 

\begin{center}[a]
\begin{tabular}{ c c c }
 cell1 & cell2 & cell3 \\ 
 cell4 & cell5 & cell6 \\  
 cell7 & cell8 & cell9    
\end{tabular}
\end{center}

\begin{center}[b]
\begin{tabular}{ |c|c|c| } 
 \hline
 cell1 & cell2 & cell3 \\ 
 cell4 & cell5 & cell6 \\ 
 cell7 & cell8 & cell9 \\ 
 \hline
\end{tabular}
\end{center}

\begin{center}[c]
 \begin{tabular}{||c c c c||} 
 \hline
 Col1 & Col2 & Col2 & Col3 \\ [0.5ex] 
 \hline\hline
 1 & 6 & 87837 & 787 \\ 
 \hline
 2 & 7 & 78 & 5415 \\
 \hline
 3 & 545 & 778 & 7507 \\
 \hline
 4 & 545 & 18744 & 7560 \\
 \hline
 5 & 88 & 788 & 6344 \\ [1ex] 
 \hline
\end{tabular}
\end{center}

\begin{center}[d]
\begin{tabular}{ | b{5cm} | m{2cm}| m{2cm} | } 
\hline
The aligning options are m for middle, p for top and b for bottom.& cell2 & cell3 \\ 
\hline
cell1 dummy text dummy text dummy text & cell5 & cell6 \\ 
\hline
cell7 & cell8 & cell9 \\ 
\hline
\end{tabular}
\end{center}

\begin{center}[e]
\begin{tabular}{ |p{3cm}||p{3cm}|p{3cm}|p{3cm}|  }
 \hline
 \multicolumn{4}{|c|}{Country List} \\
 \hline
 Country Name     or Area Name& ISO ALPHA 2 Code &ISO ALPHA 3 Code&ISO numeric Code\\
 \hline
 Afghanistan   & AF    &AFG&   004\\
 Aland Islands&   AX  & ALA   &248\\
 Albania &AL & ALB&  008\\
 Algeria    &DZ & DZA&  012\\
 American Samoa&   AS  & ASM&016\\
 Andorra& AD  & AND   &020\\
 Angola& AO  & AGO&024\\
 \hline
\end{tabular}
\end{center}


\begin{center}[f]
\begin{tabular}{ |c|c|c|c| } 
\hline
col1 & col2 & col3 \\
\hline
\multirow{3}{4em}{Multiple row} & cell2 & cell3 \\ 
& cell5 & cell6 \\ 
& cell8 & cell9 \\ 
\hline
\end{tabular}
\end{center}

\section{Detección}



\begin{table}
\footnotesize
\centering
\caption{Mauris et imperdiet     tortor. Maecenas consectetur lacus elit, dignissim eleifend dolor  ornare ut. Aenean euismod porta nisi, et volutpat ex laoreet sit amet. Sed ac elit vestibulum neque ultrices feugiat}
\label{tab:recopilacionDeCuestionarios}
%\rotatebox{90}{
\begin{tabular}{m{0.2cm}m{2.5cm}m{2.5cm}m{2.5cm}m{2.5cm}m{2.5cm}}
\hline\noalign{\smallskip}
 & \textbf{FFS} & \textbf{SOFA} & \textbf{FQ} & \textbf{CIS20R} & \textbf{FACIT}
\\ \noalign{\smallskip}
\hline
\noalign{\smallskip}
1	&	TAF	&	TAF	&	PF	&	PF	&	PF\\
2	&	TAF	&	CM	&	CS	&	EE	&	PF\\
3	&	PF	&	TAF	&	CS	&	CM	&	EE\\
\hline
\end{tabular}
%}
\end{table}



\section{Extracci\'on de caracter\'isticas}\label{secc:pruebasFisicas}


\begin{equation} \label{eq:demandaOxigeno_simple}
\begin{split} 
& K = R + H + V \\ 
& R = \textrm{consumo de oxígeno} \times kg^{-1} \times min^{-1}\\ 
& H = \textrm{constante horizontal} \times \textrm{velocidad de desplazamiento}\\ 
& V = \textrm{constante vertical} \times \textrm{velocidad de desplazamiento}\\ 
\end{split} 
\end{equation} 


\begin{figure}\label{eq:testing}
\[ e = m c^2 \]
\caption{A $ \dfrac{3}{2} $ famous equation, where $ e = m c^2  $ represent $ e = m c^2  $ energy something and $  A_{d} = -g - \frac{\sum F}{mass} $  another $\protect\pi=\protect\varpi + \protect\xi$ thing. }
\end{figure}


\section{Reconocimiento}\label{secc:disenoMetodoFatiga}



\subsection{Extracción de parámetros}
Ecuación \ref{eq:gravedad}. 

\begin{equation} \label{eq:gravedad}
A_{d} = -g - \frac{\sum F}{mass}
\end{equation} 


 


Donde $A_{d}$ representan la aceleración que se aplica a un dispositivo, $g$ la constante de gravedad de 9.81 m/$s^{2}$, y $\sum F$ las fuerzas que se aplican al propia sensor.

\subsection{Cálculo de la demanda de oxígeno} \label{secc:calculoDemandaOxigeno}
Text  \ref{eq:demandaOxigeno_personalizada} and more $O_{2}$ text:

\begin{equation} \label{eq:demandaOxigeno_personalizada}
\begin{split} 
& K = R + H + V \\ 
& R = 3.5 - (0.0367 \times BMI) - (0.0038 \times age) + (0.1790 \times gender)\\
& H = 0.1 \times \textrm{velocidad de desplazamiento}\\ 
& V = 1.8 \times \textrm{velocidad de desplazamiento}\\ 
\end{split} 
\end{equation} 

Donde $1 + 2$ representan el consumo de $O_{2}$ en reposo personalizado al usuario $(ml \times kg^{-1} \times min^{-1})$ \citep{Barstow:1991}, $H$ el componente horizontal relativo a la velocidad de desplazamiento (m/min), $V$ el componente vertical relativo a la velocidad (m/min) y pendiente de desplazamiento (\%).

\begin{description}
  \item[Velocidad:] \hfill \\
  	Para obtener la velocidad de desplazamiento se utiliza el número de pasos realizados por el usuario como se muestra a continuación (Ecuación \ref{eq:velocidad}).
  
\begin{equation} \label{eq:velocidad}
  S_{k} = D_{k}/W \hspace{10 mm} 
  D_{k} = ST_{k} \times SL \hspace{10 mm}
  SL = D_{total}/ST_{total} 
\end{equation}

\end{description}
	
\newpage
%%=====================================================

