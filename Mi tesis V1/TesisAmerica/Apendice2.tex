\appendix{}

\chapter{Algoritmo Adaboost en forma de cascada.} \label{capit:apendB}
%
\setcounter{figure}{0}
\numberwithin{figure}{chapter}
\numberwithin{table}{section} 

Enseguida se presenta el algoritmo AdaBoost en forma de cascada.  

\begin{algorithm}[h!]
\begin{algorithmic}[2]

\REQUIRE Imágenes positivas $P$, negativas $N$, $f$ el valor máximo de precisión de falsos positivos por etapa, $d$ es el valor mínimo de precisión en la detección por etapa. 
\ENSURE El clasificador en forma de cascada.  

\STATE  $F_0 = 1$, $D_0=1.$

\STATE  $i=0.$ 

\WHILE {$F_i > F_{Tarjet}$} 
	\STATE $i=i+1.$
	\STATE $n_i=0$, $F_i=F_{i-1}.$
	\WHILE {$F_I > F \times fp_{i-1}$} 
		\STATE $n_i = n_i +1.$ 
		\STATE Entrenar un clasificador usando AdaBoost con $P$, $N$ y $n_i$ características. 
		\STATE Evaluar el clasificador de cascada para determinar $F_i$ y $D_i$ en el conjunto de validación. 
		\STATE Decrementar el umbral para el $i$-ésimo clasificador hasta que el actual clasificador en cascada tenga un grado 			de detección de por lo menos $d \times D_i-1$.
	\ENDWHILE
	\STATE $N=0.$ 
	\IF {$F_i > F_{Tarjet}$} 
		\STATE Evaluar el actual clasificador en cascada en el conjunto de imágenes negativas y poner cualquier detección 				falsa en el conjunto $N$.
	\ENDIF

\ENDWHILE    
\caption{}\label{alg:Cascade} 
\end{algorithmic}
\end{algorithm} 