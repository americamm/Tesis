\chapter{Conclusiones}\label{capit:cap6}
\vspace{-2.0325ex}%
\noindent
\rule{\textwidth}{0.5pt}
\vspace{-5.5ex}% 
\newcommand{\pushline}{\Indp}% Indent puede ir o no :p

El reconocimiento de gestos basando en el modelo de la visión es un problema muy complejo ya que hay varios aspectos que se tienen que tomar en cuenta, el dispositivo de captura, la resolución de este, la variación en la intensidad de la luz del ambiente, el fondo, el tamaño de la mano, su color y la rotación que \'estas pueda presentar. Tomando en cuenta que el reconocimiento debe funcionar en tiempo real para que sea escalable a aplicaciones que pueden ser utilizadas en la vida diaria. 

En este trabajo se abord\'o el reconocimiento de gestos proponiendo una sistema que utiliza como media de captura dos dispositivos Kinect, en especifico los sensores de profundidad de estos dispositivos. La idea de utilizar estos dispositivos es que el reconocimiento de gestos funcione en condiciones bajas de iluminación y cuando exista obstrucción de la mano.  

Los resultados sugieren que, como era de esperarse, en algunas situaciones el reconocimiento mejora con la utilización de dos dispositivos. Estas situaciones son donde la mano no es vista por el Kinect principal o la información proporcionada por este dispositivo no es suficiente para discriminar el gesto.
   
Los experimentos fueron realizados en ambientes naturales, de manera que no siempre se obtuvo un alto grado de exactitud en el reconocimiento pero se lleg\'o a obtener hasta $88 \%$ para los gestos estáticos y para los dinámicos un $100 \%$ de reconocimiento en secciones de cinco segundos. Los experimentos sugieren que en situaciones controladas la tasa de reconocimiento aumentaría considerablemente. 

Debido a la realización de este trabajo se lograron las siguientes aportaciones, aparte del sistema creado: 
 
\begin{itemize}

\item Creación de una base de datos de imágenes de profundidad, de la mano y de distintos fondos.   

\item Creación de dos detectores usando el método desarrollado por Viola y Jones (2001). Uno detecta la palma de la mano con los dedos separados entre ellos. El segundo también detecta la pose anterior y dos poses más, la palma de la mano con los dedos juntos y el puño. 

\end{itemize}


%::::::::::::::::::::::::::::::::::::::::::::::::::::::::::::::::::::::


\section{Trabajo futuro}\label{futureWork}  

En la sección anterior se mencion\'o que una limitante es la resolución del sensor, una opción ser\'ia probar con la nueva versión del sensor Kinect, ya que el dispositivo tiene mayor resolución y las imágenes provenientes del sensor contienen menos ruido en comparación con la versión usada en este trabajo. 

El sistema podría mejorarse y alcanzar un mayor grado de precisión, si se mejora la detección, la propuesta es entrenar nuevamente el clasificador; incrementando el número de imágenes de entrenamiento, que contengan distintas poses, para así tener un número mayor de gestos a reconocer.

Otro punto que se puede explorar es abordar de manera distinta el reconocimiento de los gestos dinámicos, una buena propuesta seria utilizar un modelo estadístico como el Modelo Oculto de Markov, el cual permitiría implementar gestos dinámicos más complejos.  


\section{Trabajo derivado de esta tesis} 

Publicación de articulo y presentación de póster en el congreso SPIE Optics and Photonics 2015.  

\newpage
%%=====================================================