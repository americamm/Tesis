\chapter{Conclusiones}\label{capit:cap6}
\vspace{-2.0325ex}%
\noindent
\rule{\textwidth}{0.5pt}
\vspace{-5.5ex}% 
\newcommand{\pushline}{\Indp}% Indent puede ir o no :p

El objetivo del trabajo era reconocer gestos con las manos bajo distintas circunstancias. 

%Para ello se utilizo el sensor Kinect información de profundidad y  pode  se utilizaron dos dispositivos Kinect para obtener mayor información del gesto y sobrellevar problema de oclusión, gracias a esto.

\section{Limitaciones del sistema}

Esta investigación utiliza dos dispositivos Kinect como medio de entrada del sistema. De  manera que las limitaciones del sistema propuesto están dadas por las características de dicho dispositivo, tales como la distancia a la que se encuentra el dispositivo con el usuario, $0.4m.$ a $2m.$, la resolución de las imágenes a color $640 \, x \, 480$ pixeles y la resolución del sensor infrarrojo $640 \, x \, 480$ pixeles y el ruido proveniente de ambos dispositivos.

Otra limitante del sistema corresponde a las manos y por ende a los gestos que reconoce el sistema. Por ejemplo solo esta programado para detectar una mano y esta tiene que tener una ligera rotación en el eje vertical para que el sistema la detecte. El número de gestos que reconoce el sistema es limitados, pues solo reconoce dos gestos estáticos y dos dinámicos. 


\section{Aportaciones}  

Debido a la realización de este trabajo se lograron las siguientes aportaciones, aparte del sistema creado: 
 
\begin{itemize}
\item Creación de una base de datos de imágenes de profundidad, de la mano y de distintos fondos.   

\item Creación de dos detectores usando el método desarrollado por Viola y Jones (2001). Uno detecta la palma de la mano con los dedos separados entre ellos. El segundo también detecta la pose anterior y dos poses más, la palma de la mano con los dedos juntos y el puño. 

\item Publicación de articulo y presentación de póster en el congreso SPIE Optics and Photonics 2015.     

\end{itemize}


\section{Trabajo futuro}\label{futureWork}  

En la sección anterior se menciono que una limitante es la resolución del sensor, una opción seria probar con la nueva versión del sensor Kinect, ya que el dispositivo tiene mayor resolución y las imágenes provenientes del sensor contienen menos ruido en comparación con la versión anterior. 

El sistema podría mejorarse y alcanzar un mayor grado de precisión, si se mejora la detección, la propuesta es entrenar nuevamente el clasificador; incrementando el número de imágenes de entrenamiento, que contengan distintas poses, para así tener un número mayor de gestos a reconocer.

Otro punto que se puede explorar es abordar de manera distinta el reconocimiento de los gestos dinámicos, una buena propuesta seria utilizar un modelo estadístico como el Modelo Oculto de Markov, el cual permitiría implementar gestos dinámicos más complejos.  

\newpage
%%=====================================================