\appendix{}

\chapter{Algoritmo que calcula el número de dedos.} \label{aped:C}
\vspace{-3ex}%
\noindent
\rule{\textwidth}{1pt}
\vspace{-2ex}% 

A  se presenta el algoritmo para calcular el numero de dedos.  

\begin{algorithm}[h!]
\begin{algorithmic}[1]
\REQUIRE Los conjuntos $\mathcal{S}$, $\mathcal{D}$, el punto $C_r$ y el valor $L_r$.   
\ENSURE Número de dedos levantados, $Nf$.  
\FOR{$i=1$ hasta $n$}  
	\STATE $k=6$. 	
	
	\IF{$ \left[ s_i(x,y) < C_r(x,y)$ \textbf{O} $d_i(s,y) < C_r(x,y) ]$ \textbf{Y} $s_i(x,y) < d_i(x,y)$ \textbf{Y} $ \delta_i > \frac{L_r}{k}$ } 
	\STATE $Nf=Nf+1$
	\ENDIF 
\ENDFOR 

\caption{Cálculo del número de dedos levantados de la mano.}
\label{alg:NumDedos} 
\end{algorithmic}
\end{algorithm} 