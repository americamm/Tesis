%TCIDATA{Version=5.00.0.2552}
%TCIDATA{LaTeXparent=0,0,Tesis.tex}

El reconocimiento de gestos con las manos ha sido un tema relevante en distintas áreas de las ciencias de la computación, por ejemplo en HCI por sus siglas en inglés Human Computer Interaction. La relevancia de los gestos con las manos, es que con ellos, se puede obtener una interacción natural entre la computadora y el usuario, por lo que se han desarrollado diversos m\'etodos para encontrar un modelo que funcione en tiempo real y en diversas circunstancias. En este trabajo se propone un método que fusiona información proveniente de dos sensores Kinect, para realizar el reconocimiento de gestos estáticos y dinámicos en tiempo real, en circunstancias de baja iluminación y cuando existe obstrucción por parte de los dedos de la mano.