\chapter{Introducci\'on}\label{capit:cap1}
\vspace{-2.0325ex}%
\noindent
\rule{\textwidth}{0.5pt}
\vspace{-5.5ex}% 
\newcommand{\pushline}{\Indp}% Indent puede ir o no :p

La interacción entre humanos se lleva a cabo gracias a la comunicación  que existe entre ellos, esta puede ser oral o escrita, generalmente, por no decir siempre, viene acompañada de gestos realizados con la cara, manos o cuerpo. 
Estos gestos sirven como complemento de la comunicación y ayudan a que el mensaje sea percibido de manera correcta.


El creciente desarrollo de la tecnología, a llevado  a crear y estudiar distintas áreas de las ciencias computacionales, particularmente el área de interacción humano computadora (HCI, por sus siglas en ingl\'es Human Computer Interaction), la área encargada del estudio y diseño de la forma en que el humano interactua con la computadora. 
Uno de los objetivos principales de esta área es que la interacción se lleve acabo de manera natural. 
No resulta extraño que los investigadores de HCI se hayan interesado en los gestos corporales, en especial los gestos realizados con las manos, para crear un ambiente natural entre el usuario y la computadora.  
Por lo que es necesario que la computadora pueda identificar la o las manos del usuario y reconocer el gesto que este realiza. 

A finales de los años noventa se empezar\'on a desarrollar t\'ecnicas para  el reconocimiento de gestos con las manos. Los primeros acercamientos utilizaban sensores como: guantes de datos, marcadores de colores y acelerómetros; los cuales se colocaban en la o las manos para poder capturar la posición, la pose de la o las manos, entre otros datos para poder reconocer el gesto realizado. 
Las técnicas desarrolladas posteriormente obtienen la información necesaria de la mano usando distintos tipos de imágenes o vídeos, que son obtenidos mediante diversos tipos de cámaras.

Los métodos que utilizan imágenes o vídeo son los más utilizados ya que el usuario y la computadora interactuan de manera natural, el detalle con estos métodos es que es un problema difícil de resolver pues existen distintas variables que entran en juego para obtener una buena precision en el reconocimiento tomando en cuenta todas la variables. 

Aunque existe gran variedad de métodos y sistemas que hacen el reconocimiento de gestos de las manos no existe alguno que su reconocimiento tenga un alto grado de precisión en todas las situaciones que se presentan en el mundo real.  

Es por eso que se propone crear un sistema que reconozca gestos realizados con las manos, en situaciones que presentan baja iluminación y cuando existe oclusión de los dedos. 
El sistema se enfoca en atacar estos problemas cuando las manos no se encuentran en movimiento, pero también se abordar\'an los gestos con las manos que involucran movimiento. El objetivo del sistema es el de controlar la computadora mediante gestos, con la ayuda del sensor Kinect como herramienta para la captura de la información de entrada. 
  


\section{Definici\'on del problema}\label{sec:DefinicionProblema}

Existen diversas técnicas que logran obtener buena precisión en el reconocimiento de gestos realizados con las manos, pero hay técnicas que puedan tener buena precision y que se adecue a todo tipo de situaciones de la vida real como: amigable con el usuario, invariante a la iluminación, rotación, al fondo, que funcione en tiempo real o cuando exista oclusión.



\section{Justificaci\'on}\label{sec:Just}

Debido a la complejidad del problema de reconocimiento de gestos con las manos, las técnicas desarrolladas y actuales se enfocan en aspectos específicos para obtener un buen grado de precisión. De manera que se necesitan nuevos métodos que aborden los aspectos dejados de lado y funcionen no solo en condiciones ideales si no en situaciones que se presentan de manera natural y al mismo tiempo se obtenga un alto grado de precisión.  

Una vez logrado lo anterior se pueden desarrollar nuevas aplicaciones y tecnologías que ayuden a interactuar con naturalidad al usuario y la computadora.



\section{Objetivo general}\label{sec:ObjetivoGeneral}
 
Desarrollar un sistema que permita controlar la computadora haciendo uso de gestos con las manos, estáticos y dinámicos. El sistema debe ser robusto, funcionar en circunstancias de baja iluminación, cuando exista oclusión en gestos dinámicos.



\section{Objetivos espec\'ificos}\label{sec:objetivosEspecificos}

\begin{itemize}
	\item Identificar los m\'etodos actuales de reconocimiento de gestos, estáticos y din\'amicos cuando existe baja iluminación  y en el caso de los gestos dinámicos cuando existe oclusión. 
	
	\item  Obtener conocimiento acerca del funcionamiento de sistema Microsoft Kinect.
	
	\item Desarrollar un sistema de reconocimiento de gestos estáticos y dinámicos, fusionando la información de los sensores de  profundidad de dos dispositivos kinect. El sistema desarrollado deberá funcionar en circunstancias de baja iluminación y también cuando existe oclusión, causada por los dedos. 
	
	\item Analizar el sistema dise\~nado, en cuanto a su eficiencia presentada en base al reconocimiento de los gestos y tiempo de respuesta, en circunstancias de baja iluminación y oclusión. En el análisis del sistema se usar\'a información real.  
	
	\item Comparar los modelos propuestos  con los existentes, en base al tiempo de respuesta y la eficiencia en cuanto al reconocimiento del gesto. 
\end{itemize}



\section{Limitaciones y suposiciones}\label{sec:Limitaciones&Suposiciones}

Gran porcentaje de los trabajos previos en el \'area de reconocimiento de gestos con las manos basados en el modelo de la visión  utilizan c\'amaras digitales o c\'amaras web. Esta investigación utiliza dos dispositivos Kinect, para obtener la información de entrada del sistema.

De  manera que las limitaciones del sistema propuesto están dadas por las características de dicho dispositivo, tales como la distancia  a la que se encuentran los dispositivos con el usuario y la resolución del sensor. 

Otra limitante es el número de gestos que podrá reconocer el sistema.

Se supone el área de trabajo como un cuarto estándar con buena iluminación.




\section{Reconocimiento de gestos con la manos}\label{sec:ReconocimientoGestos} 

La definición de gestos \citep{Mitra2007} son movimientos del cuerpo expresivos y significativos que involucran a los dedos, manos, brazos, cabeza, cara o cuerpo con la intención de transmitir información relevante o de interactuar con el ambiente. De aquí en adelante entiéndase el término gestos con las manos, como gestos. 

Los primeros acercamientos para llevar acabo el reconocimiento de gestos fue usando modelos de contacto \citep{Rautaray2012} y \citep{Nayakwadi2014}, como su nombre lo dice utilizan dispositivos que est\'an en contacto f\'isico con la mano del usuario, esto para capturar el gesto a reconocer, por ejemplo existen guantes de datos, marcadores de colores, acelerómetros y pantallas multi-touch, aunque estos no son tan aceptados pues entorpecen la naturalidad entre la interacción del humano y la computadora. Los modelos basados en la visi\'on surgieron como respuesta a esta desventaja, estos utilizan cámaras para extraer la información necesaria para realizar el reconocimiento, los dispositivos van desde c\'amaras web hasta algunas más sofisticadas por ejemplo c\'amaras de profundidad.  

En este trabajo, se toma el enfoque basado en la visi\'on ya que se quiere obtener un sistema que para el usuario la interaccion interacción sea natural y una manera de lograr esto es tomando este enfoque.  



\section{Estado del arte}\label{sec:EstadoDelArte} 

La sección anterior explica los distintos enfoques para llevar acabo el reconocimiento de gestos, a continuación se encuentran los trabajos relevantes de cada uno de estos enfoques. 

\subsection{Modelos de contacto}

\citep{Yoon2012} propone un modelo de mezclas adaptativo, usando un guante de datos, la principal limitante para este sistema es que solo reconoce gestos estáticos. 

Aunque estos sistemas nos evitan algunos problemas que son consecuencia de los modelos basados en la visión, nos son perfectos, lo cual veremos enseguida.  

Uno de los dispositivos recientes es MYO \footnote{https://www.thalmic.com/en/myo/}, aunque de este se hablar\'a en la ultima parte de esta secci\'on. 

Como se describi\'o en la secci\'on anterior en los modelos de contacto la principal limitante es el uso de dispositivos en el cuerpo para el reconocimiento de los gestos, por esta raz\'on la mayor\'ia de los sistemas para el reconocimiento estan enfocados en modelos basados en visi\'on. Por lo que resulta natural que la investigaci\'on propuesta tome un enfoque basado en la visi\'on.

\subsection{Modelos basados en la visi\'on}  

\cite{Premaratne2013} realizan un modelo de reconocimiento de gestos est\'atico y din\'amico basados en el algoritmo de Lucas-Kanade. Las principales ventajas de este m\'etodo son que es invariante a rotaci\'on, escala y al fondo. Aunque el modelo es afectado por los cambios en la iluminaci\'on.

\citep{Huang2011}, propone un método para calculas gestos estáticos y dinámicos usando los filtros Gabor y haciendo una estimación del \'angulo en el que se encuentra la mano. Las principales ventajas son que el sistema funciona con cambios en la iluminaci\'on y es robusto a la rotaci\'on y escala. La desventaja es que el problema de oclusi\'on no es tratado.

\citep{MohdAsaari2014} hacen el seguimiento de la mano para identificar los gestos din\'amicos usando los filtros adaptativos Kalman y el m\'etodo Eigenhand. Con esta combinaci\'on obtienen un excelente resultado pues el sistema es robusto a la iluminaci\'on, cambio de pose, y a la oclusi\'on causada la mano oculta por alg\'un objeto en movimiento. 

 

A pesar que la mayoría de los modelos vistos en la parte de arriba solucionan muchos de los problemas de los modelos basados en la visi\'on. Ninguno de ellos puede resolver el problema de iluminaci\'on y oclusi\'on, formada por lo dedos. All\'i la importancia de la investigaci\'on propuesta, pues dar\'a soluci\'on a estos inconvenientes al momento de reconocer los gestos.
 
\subsection{Sistemas comerciales}

Existen dispositivos como: Leap Motion \footnote{https://www.leapmotion.com/}, MYO, y software, como Flutter \footnote{https://flutterapp.com/}, que realizan el reconocimiento de gestos, y estos los utilizan como reemplazo del rat\'on de la computadora. 
 
Leap Motion es un dispositivo que detecta los movimientos de manos y dedos por medio de sensores infrarrojos. Leap Motion es robusto con el fondo, escala y rotaci\'on,  pero no cuando existe oclusi\'on pues cuando se realiza un zoom, como el que se hace en cualquier dispositivo touch, produce un error, y se presenta cuando un dedo es cubierto por otro, un problema grave es que tiene problemas de reconocimiento en circunstancias normales de luz.  

MYO este dispositivo, solo se encuentra en pre-ordenamiento, detecta los impulsos el\'ectricos de tus m\'usculos mediante tres sensores, giroscopio, aceler\'ometro y magnet\'ometros. MYO es un brazalete  que promete controlar la computadora y dispositivos tales como el celular o la tableta. La principal desventaja del sensor es que gestos involuntarios pueden producir acciones no deseadas.

Flutter  es un software que reconoce cuatro gestos est\'aticos detectando la palma de la mano, usando la c\'amara web como dispositivo de entrada. Flutter permite controlar aplicaciones multimedia de la computadora. 	Las limitaciones del software son que solo reconoce gestos est\'aticos, realiza acciones no deseadas al hacer gestos involuntarios y no siempre reconoce los gestos. 

Aunque estos dispositivos y software para reconocer gestos  solucionan algunos problemas importantes en el \'area, sigue existiendo el problema de oclusi\'on  e iluminaci\'on. De allí la importancia que existan modelos que puedan resolver estos problemas se presentan frecuentemente en el reconocimiento. 

\section{Organizaci\'on de la tesis}\label{sec:OrganizacionTesis}
La tesis se encuentra distribuida de la siguiente manera: la segunda sección presenta los fundamentos teóricos como base para la comprensión del tema. La tercera sección presenta el sistema propuesto. La cuarta sección se encuentra a detalle la implementación del sistema. En la quinta sección se presentan las pruebas realizadas al sistema junto con los resultados y la discusiones de estos. Finalmente la sexta sección presenta las conclusiones generales del sistema y el trabajo futuro. 
	


%This text is normal size, but
%\large
%all of this text is pretty big!
%Except for a couple
%{\Large really} big
%{\LARGE obnoxiously} large
%{\Huge words}.
%You can get back down to size with some
%{\normalsize normal text}.
%If you're feeling
%{\small small}, you can get really
%{\tiny tiny}, but you can also imitate other sizes like
%{\footnotesize footnotes} and
%{\scriptsize subscripts and superscripts}

\newpage
%%=====================================================









