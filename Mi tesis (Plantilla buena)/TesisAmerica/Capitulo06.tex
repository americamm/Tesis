\chapter{Conclusiones}\label{capit:cap6}
\vspace{-2.0325ex}%
\noindent
\rule{\textwidth}{0.5pt}
\vspace{-5.5ex}% 
\newcommand{\pushline}{\Indp}% Indent puede ir o no :p


%\begin{figure}[!hbp]
%\begin{center}
%\includegraphics[scale=1.5]
%{./Figures/doge.jpeg}
%\end{center}
%\caption{Wow sdf.}
%\label{doge}
%\end{figure}

\section{Limitaciones del sistema}

Gran porcentaje de los trabajos previos en el \'area de reconocimiento de gestos con las manos basados en el modelo de la visión  utilizan c\'amaras digitales o c\'amaras web. Esta investigación utiliza el dispositivo Kinect, para obtener la información de entrada del sistema. 

De  manera que las limitaciones del sistema propuesto están dadas por las características de dicho dispositivo, tales como la distancia a la que se encuentra el dispositivo con el usuario, $0.4m$ a $3m$ , la resolución de las imágenes a color $640 \, x \, 480$ pixeles y la resolución del sensor infrarrojo $640 \, x \, 480$ pixeles.\\
También el sistema depende de dos sensores Kinect, que se utilizar\'an en el caso que exista oclusión. \\
Otra limitante es el número de gestos que podrá reconocer el sistema.

Se supone el área de trabajo como un cuarto estándar con buena iluminación (enfocado a pruebas con la cámara color del sistema Kinect).

\section{Trabajo futuro}\label{futureWork} 
\newpage
%%=====================================================