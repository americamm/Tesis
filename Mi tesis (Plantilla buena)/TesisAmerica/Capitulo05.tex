\chapter{Resultados}\label{capit:cap5}
\vspace{-2.0325ex}%
\noindent
\rule{\textwidth}{0.5pt}
\vspace{-5.5ex}% 
\newcommand{\pushline}{\Indp}% Indent puede ir o no :p

El sistema propuesto fue implementado en una computadora de escritorio Dell con un procesador Intel(R) Xeon(R) CPU E5-1603, 16GB de memoria RAM, Windows 7 de 64 bits. La implementación del sistema se realizó en C\# utilizando Emgu 2.410 \footnote{http://www.emgu.com/wiki/index.php/Main\_Page} un wrapper de OpenCV\footnote{http://opencv.org/}.  

Para probar la precisión del sistema se realizaron diversas pruebas con los distintos tipos de gestos y en diferentes circunstancias. En las secciones siguientes se explica cada experimento y resultados de estos.  

\section{Experimentos de gestos estáticos}\label{TestStaticGestures} 

\section{Experimentos de gestos dinámicos}\label{TestDinamicGestures} 
 
\newpage
%%=====================================================
%\begin{tabular}{ c c c }
% cell1 & cell2 & cell3 \\ 
% cell4 & cell5 & cell6 \\  
% cell7 & cell8 & cell9    
%\end{tabular}
%\end{center}
%
%\begin{center}[b]
%\begin{tabular}{ |c|c|c| } 
% \hline
% cell1 & cell2 & cell3 \\ 
% cell4 & cell5 & cell6 \\ 
% cell7 & cell8 & cell9 \\ 
% \hline
%\end{tabular}
%\end{center}
%
%\begin{center}[c]
% \begin{tabular}{||c c c c||} 
% \hline
% Col1 & Col2 & Col2 & Col3 \\ [0.5ex] 
% \hline\hline
% 1 & 6 & 87837 & 787 \\ 
% \hline
% 2 & 7 & 78 & 5415 \\
% \hline
% 3 & 545 & 778 & 7507 \\
% \hline
% 4 & 545 & 18744 & 7560 \\
% \hline
% 5 & 88 & 788 & 6344 \\ [1ex] 
% \hline
%\end{tabular}
%\end{center}
%
%\begin{center}[d]
%\begin{tabular}{ | b{5cm} | m{2cm}| m{2cm} | } 
%\hline
%The aligning options are m for middle, p for top and b for bottom.& cell2 & cell3 \\ 
%\hline
%cell1 dummy text dummy text dummy text & cell5 & cell6 \\ 
%\hline
%cell7 & cell8 & cell9 \\ 
%\hline
%\end{tabular}
%\end{center}
%
%\begin{center}[e]
%\begin{tabular}{ |p{3cm}||p{3cm}|p{3cm}|p{3cm}|  }
% \hline
% \multicolumn{4}{|c|}{Country List} \\
% \hline
% Country Name     or Area Name& ISO ALPHA 2 Code &ISO ALPHA 3 Code&ISO numeric Code\\
% \hline
% Afghanistan   & AF    &AFG&   004\\
% Aland Islands&   AX  & ALA   &248\\
% Albania &AL & ALB&  008\\
% Algeria    &DZ & DZA&  012\\
% American Samoa&   AS  & ASM&016\\
% Andorra& AD  & AND   &020\\
% Angola& AO  & AGO&024\\
% \hline
%\end{tabular}
%\end{center}
%\begin{center}[f]
%\begin{tabular}{ |c|c|c|c| } 
%\hline
%col1 & col2 & col3 \\
%\hline
%\multirow{3}{4em}{Multiple row} & cell2 & cell3 \\ 
%& cell5 & cell6 \\ 
%& cell8 & cell9 \\ 
%\hline
%\end{tabular}
%\end{center} 


%\begin{table}
%\footnotesize
%\centering
%\caption{Mauris et imperdiet     tortor. Maecenas consectetur lacus elit, dignissim eleifend dolor  ornare ut. Aenean euismod porta nisi, et volutpat ex laoreet sit amet. Sed ac elit vestibulum neque ultrices feugiat}
%\label{tab:recopilacionDeCuestionarios}
%%\rotatebox{90}{
%\begin{tabular}{m{0.2cm}m{2.5cm}m{2.5cm}m{2.5cm}m{2.5cm}m{2.5cm}}
%\hline\noalign{\smallskip}
% & \textbf{FFS} & \textbf{SOFA} & \textbf{FQ} & \textbf{CIS20R} & \textbf{FACIT}
%\\ \noalign{\smallskip}
%\hline
%\noalign{\smallskip}
%1	&	TAF	&	TAF	&	PF	&	PF	&	PF\\
%2	&	TAF	&	CM	&	CS	&	EE	&	PF\\
%3	&	PF	&	TAF	&	CS	&	CM	&	EE\\
%\hline
%\end{tabular}
%%}
%\end{table}