\chapter{Resultados}\label{capit:cap5}
\vspace{-2.0325ex}%
\noindent
\rule{\textwidth}{0.5pt}
\vspace{-5.5ex}% 
\newcommand{\pushline}{\Indp}% Indent puede ir o no :p

%Sobre referencias. CICESE pide este formato \citep{Adleman1998}


\newpage
%%=====================================================
%\begin{tabular}{ c c c }
% cell1 & cell2 & cell3 \\ 
% cell4 & cell5 & cell6 \\  
% cell7 & cell8 & cell9    
%\end{tabular}
%\end{center}
%
%\begin{center}[b]
%\begin{tabular}{ |c|c|c| } 
% \hline
% cell1 & cell2 & cell3 \\ 
% cell4 & cell5 & cell6 \\ 
% cell7 & cell8 & cell9 \\ 
% \hline
%\end{tabular}
%\end{center}
%
%\begin{center}[c]
% \begin{tabular}{||c c c c||} 
% \hline
% Col1 & Col2 & Col2 & Col3 \\ [0.5ex] 
% \hline\hline
% 1 & 6 & 87837 & 787 \\ 
% \hline
% 2 & 7 & 78 & 5415 \\
% \hline
% 3 & 545 & 778 & 7507 \\
% \hline
% 4 & 545 & 18744 & 7560 \\
% \hline
% 5 & 88 & 788 & 6344 \\ [1ex] 
% \hline
%\end{tabular}
%\end{center}
%
%\begin{center}[d]
%\begin{tabular}{ | b{5cm} | m{2cm}| m{2cm} | } 
%\hline
%The aligning options are m for middle, p for top and b for bottom.& cell2 & cell3 \\ 
%\hline
%cell1 dummy text dummy text dummy text & cell5 & cell6 \\ 
%\hline
%cell7 & cell8 & cell9 \\ 
%\hline
%\end{tabular}
%\end{center}
%
%\begin{center}[e]
%\begin{tabular}{ |p{3cm}||p{3cm}|p{3cm}|p{3cm}|  }
% \hline
% \multicolumn{4}{|c|}{Country List} \\
% \hline
% Country Name     or Area Name& ISO ALPHA 2 Code &ISO ALPHA 3 Code&ISO numeric Code\\
% \hline
% Afghanistan   & AF    &AFG&   004\\
% Aland Islands&   AX  & ALA   &248\\
% Albania &AL & ALB&  008\\
% Algeria    &DZ & DZA&  012\\
% American Samoa&   AS  & ASM&016\\
% Andorra& AD  & AND   &020\\
% Angola& AO  & AGO&024\\
% \hline
%\end{tabular}
%\end{center}
%\begin{center}[f]
%\begin{tabular}{ |c|c|c|c| } 
%\hline
%col1 & col2 & col3 \\
%\hline
%\multirow{3}{4em}{Multiple row} & cell2 & cell3 \\ 
%& cell5 & cell6 \\ 
%& cell8 & cell9 \\ 
%\hline
%\end{tabular}
%\end{center}