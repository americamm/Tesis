\chapter{Objetivos}

\section{Objetivo general}

Desarrollar un sistema que permita controlar la computadora haciendo uso de gestos con las manos, est�ticos y din�micos. El sistema debe ser robusto, funcionar en circunstancias de baja iluminaci�n, cuando exista oclusi�n en gestos din�micos.

\section{Objetivos espec\'ificos}

\begin{enumerate}
	\item Identificar los m\'etodos actuales de reconocimiento de gestos, est\'aticos y din\'amicos cuando existe baja iluminaci�n  y en el caso de los gestos din�micos cuando existe oclusi�n. 
	\item  Obtener conocimiento acerca del funcionamiento de sistema Microsoft Kinect.
	\item Desarrollar un sistema de reconocimiento de gestos est�ticos y din�micos, fusionando la informaci�n de los sensores de  profundidad del dispositivo kinect. El sistema desarrollado deber� funcionar en circunstancias de baja iluminaci�n y tambi�n cuando existe oclusi�n, causada por los dedos. 
	\item Analizar el sistema dise\~nado, en cuanto a su eficiencia presentada en base al reconocimiento de los gestos y tiempo de respuesta, en circunstancias de baja iluminaci�n y oclusi�n. En el an�lisis del sistema se usara informaci�n real.  
	\item Comparar los modelos propuestos  con los existentes, en base al tiempo de respuesta y la eficiencia en cuanto al reconocimiento del gesto. 
\end{enumerate} 


\section{Preguntas de investigaci\'on}

\begin{enumerate}
	\item >Cu�l es el desempe�o de las t�cnicas desarrolladas al compararlas con las t�cnicas cl�sicas?
	\item >Es posible mejorar el desempe�o de las t�cnicas cl�sicas usando informaci�n de mapas de profundidad multi-vista?
	\item >Es posible garantizar un grado de confiabilidad de los resultados?
\end{enumerate}