\chapter{Metodolog\'ia de la investigaci\'on}

\begin{enumerate}
	\item Realizar una revisi\'on bibliogr\'afica de los m\'etodos existentes de reconocimiento de gestos e identificar los principales m�todos con base en los problemas que atacan. 
	\item Conocer el funcionamiento del dispositivo Microsoft Kinect  y realizar algunas pruebas con los sensores y el programa.
	\item Estudiar  y comprender los modelos basados en visi\'on usados para el reconocimiento de gestos, los cuales fusionan diferentes tipos de informaci\'on de entrada. 
	\item Crear una base de datos con los gestos necesarios para el sistema. 
	\item Implementar los modelos relevantes de fusi�n, estudiados previamente.  
	\item Generaci�n de un mapa de profundidad de dos mapas de profundidad, el mapa generado deber� contar con mayor informaci�n - raz�n de se�al a ruido (SNR) de la escena.
	\item Desarrollar un sistema de reconocimiento de gestos est�ticos y din�mico utilizando informaci�n de los sensores de profundidad.
	\item Probar y evaluar  los modelos desarrollados en circunstancias distintas de iluminaci�n en base al tiempo de respuesta, a la precisi\'on del reconocimiento del gesto. La informaci�n para la evaluaci�n del sistema ser�n muestras reales.   
	\item Analizar y evaluar los modelos existentes conforme al tiempo de respuesta y la precisi\'on del reconocimiento del gesto.
\end{enumerate}




