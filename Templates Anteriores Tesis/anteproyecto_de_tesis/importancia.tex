\chapter{Importancia de la investigaci\'on}

La mayor�a de los sistemas que hacen reconocimiento de gestos con las manos no hacen uso de informaci�n 3D, mientras que los que si lo hacen no toman en cuenta el gran potencial que se encuentra en la mezcla de informaci�n 3D de varios puntos de visi�n. De manera que la importancia de este trabajo consiste en explorar nuevas m�todos basados en la utilizaci�n de informaci�n 3D obtenida de dos sensores de profundidad que recuperan informaci�n de una misma escena.

La HCI, se dedica a que la interacci\'on entre el usuario y la computadora sea de forma natural  y amigable. Y una manera de lograr u obtener un acercamiento a esta interacci\'on es con el reconocimiento de gestos,  en especial el de las manos. Aunque existen diversos m\'etodos para realizar el reconocimiento, estos funcionan en distintas circunstancias como los son el fondo est\'atico, fondo complejo, baja iluminaci\'on, oclusi\'on del gesto. \citep{Sgouropoulos2013}, \citep{Murthy2009}

Realidad aumentada, el mouse y el teclado no siempre pueden transmitir la informaci\'on necesaria cuando se trata de obtener un modelo 3D en la computadora por eso es necesario encontrar una manera de transmitir esta informaci\'on y que mejor que con el uso de un espacio 3D por lo que en este caso es de gran utilidad el reconocimiento de gestos con las manos. \citep{Reifinger2007} 
