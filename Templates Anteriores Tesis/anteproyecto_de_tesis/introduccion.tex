\chapter{Introducci�n} 

Los humanos para expresarnos utilizamos la escritura y el habla pero generalmente para complementar  usamos como complemento de la expresi�n los gestos corporales de cara, manos, etc. De manera que no es extra\~no que los investigadores de HCI se hayan interesado en los gestos corporales, en especial los gestos realizados con las manos, para crear un ambiente natural entre el usuario y la computadora. 
Pero para realizar esta interacci\'on natural se necesita hacer el reconocimiento de los gestos, esto ha sido cada vez m\'as sencillo gracias al avance de la tecnolog\'ia, en especial con  dispositivos de visi\'on como distintos tipos de c\'amaras, y al crecimiento del procesamiento de las computadoras. Aunque existen diversos m\'etodos para lograr el reconocimiento, no hay alguno que nos pueda dar un reconocimiento totalmente preciso en todas las situaciones que se presentan en el mundo  real.  


La interacci\'on del humano y la computadora (HCI, por sus siglas en \'ingles) es una rama de las ciencias computacionales dedicada al estudio, dise\~no e interacci\'on de un humano con la computadora. El objetivo principal es que, para el humano la interacci\'on entre ellos sea natural.  

Los humanos para expresarnos utilizamos la escritura y el habla pero siempre se utiliza como complemento de la expresi�n los gestos corporales de cara, manos, etc. De manera que no es extra\~no que los investigadores de HCI se hayan interesado en los gestos corporales, en especial los gestos realizados con las manos, para crear un ambiente natural entre el usuario y la computadora. 
Pero para realizar esta interacci\'on natural se necesita hacer el reconocimiento de los gestos, esto ha sido cada vez m\'as sencillo gracias al avance de la tecnolog\'ia, en especial con  dispositivos de visi\'on como distintos tipos de c\'amaras, y al crecimiento del procesamiento de las computadoras. Aunque existen diversos m\'etodos para lograr el reconocimiento, no hay alguno que nos pueda dar un reconocimiento totalmente preciso en todas las situaciones que se presentan en el mundo  real.  

Es por eso que se propone crear un sistema que haga el reconocimiento de gestos con las manos, en situaciones donde existe baja iluminaci\'on y cuando tenemos oclusi\'on causada por los dedos. El sistema se enfoca mayormente  a solucionar los problemas de gestos con las manos que no tienen movimiento, y despu\'es se abordar\'an los gestos con las manos que involucran movimiento. El sistema aplicara los gestos como control de la computadora, esto con ayuda del dispositivo Kinect como  herramienta para capturar la informaci\'on de entrada. 

La propuesta se encuentra dividida por secciones, enseguida se muestra la distibuci\'on de  cada una de estas: la segunda secci\'on se concentra en el marco te\'orico, la tercera muestra la investigaci\'on previa relevante, en la cuarta la importancia de la investigaci\'on, la quinta las limitaciones y suposiciones, la sexta contiene la contribuci\'on al conocimiento, la s\'eptima los objetivos, la octava la metodolog\'ia de la investigaci\'on y por \'ultimo la novena secci\'on  muestra el calendario de actividades que se lleva acabo a lo largo de la investigaci\'on.  