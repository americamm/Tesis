\chapter{Introducci\'on}\label{capit:cap1}
\vspace{-2.0325ex}%
\noindent
\rule{\textwidth}{0.5pt}
\vspace{-5.5ex}% 
\newcommand{\pushline}{\Indp}% Indent puede ir o no :p

La interacci\'on entre humanos se lleva a cabo gracias a la comunicaci\'on  que existe entre ellos, esta puede ser oral o escrita, generalmente, por no decir siempre, viene acompa\~nada de gestos realizados con la cara, manos, cuerpo. 
Estos gestos sirven como complemento de la comunicaci\'on ya ayudan a que nuestra idea se percibida de manera correcta.


El creciente desarrollo de la tecnolog\'ia, a llevado  a crear y estudiar distintas \'areas de las ciencias computacionales, particularmente HCI (por sus siglas en \'ingles), la \'area encarga del estudio, dise\~no  e interacci\'on del humano con la computadora. 
Uno de los objetivos principales es que la interacci\'on sea de manera natural. 
Por lo que no es extra\~no que los investigadores de HCI se hayan interesado en los gestos corporales, en especial los gestos realizados con las manos, para crear un ambiente natural entre el usuario y la computadora. 
Para obtener una interacci\'on natural, entre estos dos actuadores, se necesita hacer el reconocimiento de los gestos, esto ha sido cada vez m\'as sencillo gracias al avance de la tecnolog\'ia, en especial en los dispositivos de visi\'on como distintos tipos de c\'amaras, y al crecimiento en la capacidad de procesamiento de las computadoras. 
Aunque existen diversos m\'etodos y sistemas para lograr el reconocimiento, no existe ninguno que nos pueda dar un reconocimiento totalmente preciso en todas las situaciones que se presentan en el mundo real.  


Es por eso que se propone crear un sistema que haga el reconocimiento de gestos con las manos, en situaciones donde existe baja iluminación y cuando tenemos oclusi\'on causada por los dedos. 
El sistema se enfoca principalmente en atacar los problemas de gestos con las manos que no tienen movimiento, y despu\'es se abordar\'an los gestos con las manos que involucran movimiento. El sistema aplicar\'a los gestos como control de la computadora, esto con ayuda del dispositivo Kinect como  herramienta para capturar la informaci\'on de entrada. 
  

\section{Definici\'on del problema}\label{DefinicionProblema}

A finales de los an\~os noventa se empezar\'on a desarrollar t\'ecnicas para reconocer gestos con las manos, las primeras fueron basadas en contacto y le siguieron las basadas en la visión, estas fueron las m\'as aceptadas debido a la facilidad de interacción entre el usuario, entre otras cosas, aunque estas tienen sus desventajas pues no es problema fácil de resolver debido a que existen distintas variables a considerar.

Aunque existen diversos m\'etodos para el reconocimiento de gestos con las manos, con buena precisi\'on, sigue siendo un problema abierto ya que no es f\'acil tener un sistema que se adecue a todo tipo de situaciones como: amigable con el usuario, invariante a la iluminación, rotación, al fondo, que funcione en tiempo real o cuando exista oclusión.


\section{Justificaci\'on}\label{Just}



\section{Objetivo general}\label{ObjetivoGeneral}
 
Desarrollar un sistema que permita controlar la computadora haciendo uso de gestos con las manos, estáticos y dinámicos. El sistema debe ser robusto, funcionar en circunstancias de baja iluminación, cuando exista oclusión en gestos dinámicos.

\section{Objetivos especificos}\label{bjetivosEspecificos}

\begin{itemize}
	\item Identificar los m\'etodos actuales de reconocimiento de gestos, estáticos y din\'amicos cuando existe baja iluminación  y en el caso de los gestos dinámicos cuando existe oclusión. 
	
	\item  Obtener conocimiento acerca del funcionamiento de sistema Microsoft Kinect.
	
	\item Desarrollar un sistema de reconocimiento de gestos estáticos y dinámicos, fusionando la información de los sensores de  profundidad de dos dispositivo kinect. El sistema desarrollado deberá funcionar en circunstancias de baja iluminación y también cuando existe oclusión, causada por los dedos. 
	
	\item Analizar el sistema dise\~nado, en cuanto a su eficiencia presentada en base al reconocimiento de los gestos y tiempo de respuesta, en circunstancias de baja iluminación y oclusión. En el análisis del sistema se usar\'a información real.  
	
	\item Comparar los modelos propuestos  con los existentes, en base al tiempo de respuesta y la eficiencia en cuanto al reconocimiento del gesto. 
\end{itemize}


\section{Limitaciones y suposiciones}\label{Limitaciones&Suposicione}

Gran porcentaje de los trabajos previos en el \'area de reconocimiento de gestos con las manos basados en el modelo de la visi\'on  utilizan c\'amaras digitales o c\'amaras web. Esta investigaci\'on utiliza el dispositivo Kinect, para obtener la informaci\'on de entrada del sistema. 

De  manera que las limitaciones del sistema propuesto est\'an dadas por las características dicho dispositivo, tales como la distancia a la que se encuentra el dispositivo con el usuario (poner la distancia), la resoluci\'on de las im\'agenes a color (poner resolucion) y la resoluci\'on del sensor infrarrojo (poner resolucion).\\
Tambi\'en el sistema depende de dos sensores Kinect, que se utilizar\'an en el caso que exista oclusi\'on. \\
Otra limitante es el número de gestos que podr\'a reconocer el sistema.

Se supone el área de trabajo como un cuarto estándar con buena iluminación (enfocado a pruebas con la cámara color del sistema Kinect)

\section{Reconocimiento de gestos con la manos}\label{ReconocimientoGestos} 


La definición de gestos \citep{Mitra2007} son movimientos del cuerpo expresivos y significativos que involucran a los dedos, manos, brazos, cabeza, cara o cuerpo con la intención de transmitir información relevante o de interactuar con el ambiente. De acuerdo con la literatura \citep{Mitra2007} los gestos con las manos se clasifican en estáticos y dinámicos, los primeros están definidos como la posición y orientación de la mano en el espacio manteniendo esta pose durante cierto tiempo, por ejemplo para hacer una se\~nal de aventón, a diferencia de los gestos dinámicos donde hay movimiento de la pose, un ejemplo  es cuando mueves la mano en se\~nal de adiós. De aquí en adelante entiéndase el término gestos con las manos, como gestos. 

El reconocimiento de gestos se divide en tres fases\cite{Rautaray2012}, detección o segmentación; extracci\'on de caracter\'isticas seguimiento; dependiendo si los gestos son dinámicos, por último la etapa final el reconocimiento del gesto.  
Este se clasifican en dos modelos, basados en la visi\'on y en contacto, esta clasificaci\'on depende de la manera en que son capturados los datos, es decir la forma en que se obtiene el gesto, para posteriormente poderlo reconocer. 

Los primeros acercamientos para llevar acabo el reconocimiento de gestos fue usando modelos de contacto \cite{Rautaray2012} y \cite{Nayakwadi2014}, como su nombre lo dice utilizan dispositivos que est\'an en contacto f\'isico con la mano del usuario, esto para capturar el gesto a reconocer, por ejemplo existen guantes de datos, marcadores de colores, acelerómetros y pantallas multi-touch, aunque estos no son tan aceptados pues entorpecen la naturalidad entre la interacción del humano y la computadora. Los modelos basados en la visi\'on surgieron como respuesta a esta desventaja, estos utilizan cámaras para extraer la información necesaria para realizar el reconocimiento, los dispositivos van desde c\'amaras web hasta algunas más sofisticadas por ejemplo c\'amaras de profundida.  

En este trabajo, se toma el enfoque basado en la visi\'on ya que se quiere obtener un sistema que para el usuario sea facil de interactuar, y esta interacci\'on sea natural y una manera de lograr esto es tomando este enfoque.  estos tienen mayor complejidad (acomodra este parrafo :P)


Los métodos basados en la visión se pueden representar por dos modelos \citep{Rautaray2012}, los basados en 3D, da una descripción espacial en 3D de la mano, y los basados en apariencia, como su nombre lo dice se basan en la apariencia de la mano. Los modelos basados en apariencia se dividen en dos categorías, los estáticos (modelo de silueta, de contorno deformables) y de movimiento (de color y movimiento).


\section{Estado del arte}\label{EstadoDelArte} 

La sección anterior explica los distintos enfoques para llevar acabo el reconocimiento de gestos, a continuación se encuentran los trabajos relevantes de cada uno de estos enfoques. 

\subsection{Modelos de contacto}

\citep{Yoon2012} propone un modelo de mezclas adaptativo, usando un guante de datos, la principal limitante para este sistema es que solo reconoce gestos estáticos. 

Aunque estos sistemas nos evitan algunos problemas que son consecuencia de los modelos basados en la visión, nos son perfectos, lo cual veremos enseguida.  

Uno de los dispositivos recientes es MYO \footnote{https://www.thalmic.com/en/myo/}, aunque de este se hablar\'a en la ultima parte de esta secci\'on. 

Como se describi\'o en la secci\'on anterior en los modelos de contacto la principal limitante es el uso de dispositivos en el cuerpo para el reconocimiento de los gestos, por esta raz\'on la mayor\'ia de los sistemas para el reconocimiento estan enfocados en modelos basados en visi\'on. Por lo que resulta natural que la investigaci\'on propuesta tome un enfoque basado en la visi\'on.

\subsection{Modelos basados en la visi\'on}  

\cite{Premaratne2013} realizan un modelo de reconocimiento de gestos est\'atico y din\'amico basados en el algoritmo de Lucas-Kanade. Las principales ventajas de este m\'etodo son que es invariante a rotaci\'on, escala y al fondo. Aunque el modelo es afectado por los cambios en la iluminaci\'on.

\citep{Huang2011}, propone un método para calculas gestos estáticos y dinámicos usando los filtros Gabor y haciendo una estimación del \'angulo en el que se encuentra la mano. Las principales ventajas son que el sistema funciona con cambios en la iluminaci\'on y es robusto a la rotaci\'on y escala. La desventaja es que el problema de oclusi\'on no es tratado.

\citep{MohdAsaari2014} hacen el seguimiento de la mano para identificar los gestos din\'amicos usando los filtros adaptativos Kalman y el m\'etodo Eigenhand. Con esta combinaci\'on obtienen un excelente resultado pues el sistema es robusto a la iluminaci\'on, cambio de pose, y a la oclusi\'on causada la mano oculta por alg\'un objeto en movimiento. 

 

A pesar que la mayoría de los modelos vistos en la parte de arriba solucionan muchos de los problemas de los modelos basados en la visi\'on. Ninguno de ellos puede resolver el problema de iluminaci\'on y oclusi\'on, formada por lo dedos. All\'i la importancia de la investigaci\'on propuesta, pues dar\'a soluci\'on a estos inconvenientes al momento de reconocer los gestos.
 
\subsection{Sistemas comerciales}

Existen dispositivos como: Leap Motion \footnote{https://www.leapmotion.com/}, MYO, y software, como Flutter \footnote{https://flutterapp.com/}, que realizan el reconocimiento de gestos, y estos los utilizan como reemplazo del rat\'on de la computadora. 
 
Leap Motion es un dispositivo que detecta los movimientos de manos y dedos por medio de sensores infrarrojos. Leap Motion es robusto con el fondo, escala y rotaci\'on,  pero no cuando existe oclusi\'on pues cuando se realiza un zoom, como el que se hace en cualquier dispositivo touch, produce un error, y se presenta cuando un dedo es cubierto por otro, un problema grave es que tiene problemas de reconocimiento en circunstancias normales de luz.  

MYO este dispositivo, solo se encuentra en pre-ordenamiento, detecta los impulsos el\'ectricos de tus m\'usculos mediante tres sensores, giroscopio, aceler\'ometro y magnet\'ometros. MYO es un brazalete  que promete controlar la computadora y dispositivos tales como el celular o la tableta. La principal desventaja del sensor es que gestos involuntarios pueden producir acciones no deseadas.

Flutter  es un software que reconoce cuatro gestos est\'aticos detectando la palma de la mano, usando la c\'amara web como dispositivo de entrada. Flutter permite controlar aplicaciones multimedia de la computadora. 	Las limitaciones del software son que solo reconoce gestos est\'aticos, realiza acciones no deseadas al hacer gestos involuntarios y no siempre reconoce los gestos. 

Aunque estos dispositivos y software para reconocer gestos  solucionan algunos problemas importantes en el \'area, sigue existiendo el problema de oclusi\'on  e iluminaci\'on. De allí la importancia que existan modelos que puedan resolver estos problemas se presentan frecuentemente en el reconocimiento. 

\section{Organizaci\'on de la tesis}\label{OrganizacionTesis}
La tesis se encuentra distribuida de la siguiente manera: la segunda sección presenta los fundamentos teóricos como base para la comprensión del tema. La tercera sección presenta el sistema propuesto. En la cuarta sección se encuentran las pruebas realizadas al sistema junto con los resultados y la discusiones de estos. Finalmente la quinta sección presenta las conclusiones generales del sistema y el trabajo futuro. 
	


%This text is normal size, but
%\large
%all of this text is pretty big!
%Except for a couple
%{\Large really} big
%{\LARGE obnoxiously} large
%{\Huge words}.
%You can get back down to size with some
%{\normalsize normal text}.
%If you're feeling
%{\small small}, you can get really
%{\tiny tiny}, but you can also imitate other sizes like
%{\footnotesize footnotes} and
%{\scriptsize subscripts and superscripts}

\newpage
%%=====================================================









