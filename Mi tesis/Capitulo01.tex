\chapter{Introducci\'on}\label{capit:cap1}
\vspace{-2.0325ex}%
\noindent
\rule{\textwidth}{0.5pt}
\vspace{-5.5ex}% 
\newcommand{\pushline}{\Indp}% Indent puede ir o no :p

La interacci\'on entre humanos se lleva a cabo gracias a la comunicaci\'on  que existe entre ellos, esta puede ser oral o escrita, generalmente, por no decir siempre, viene acompa\~nada de gestos realizados con la cara, manos, cuerpo. 
Estos gestos sirven como complemento de la comunicaci\'on ya ayudan a que nuestra idea se percibida de manera correcta.


El creciente desarrollo de la tecnolog\'ia, a llevado  a crear y estudiar distintas \'areas de las ciencias computacionales, particularmente HCI (por sus siglas en \'ingles), la \'area encarga del estudio, dise\~no  e interacci\'on del humano con la computadora. 
Uno de los objetivos principales es que la interacci\'on sea de manera natural. 
Por lo que no es extra\~no que los investigadores de HCI se hayan interesado en los gestos corporales, en especial los gestos realizados con las manos, para crear un ambiente natural entre el usuario y la computadora. 
Para obtener una interacci\'on natural, entre estos dos actuadores, se necesita hacer el reconocimiento de los gestos, esto ha sido cada vez m\'as sencillo gracias al avance de la tecnolog\'ia, en especial en los dispositivos de visi\'on como distintos tipos de c\'amaras, y al crecimiento en la capacidad de procesamiento de las computadoras. 
Aunque existen diversos m\'etodos y sistemas para lograr el reconocimiento, no existe ninguno que nos pueda dar un reconocimiento totalmente preciso en todas las situaciones que se presentan en el mundo real.  


Es por eso que se propone crear un sistema que haga el reconocimiento de gestos con las manos, en situaciones donde existe baja iluminaci\'on y cuando tenemos oclusi\'on causada por los dedos. 
El sistema se enfoca principalmente en atacar los problemas de gestos con las manos que no tienen movimiento, y despu\'es se abordar\'an los gestos con las manos que involucran movimiento. El sistema aplicar\'a los gestos como control de la computadora, esto con ayuda del dispositivo Kinect como  herramienta para capturar la informaci\'on de entrada. 
  

\section{Definici\'on del problema}\label{sec: Definici\'on del problema}

nbkjghgjhyfkhfiy

\section{Justificaci\'on}\label{sec: Justificaci\'on}

A finales de los an\~os noventa se empezar\'on a desarrollar t\'ecnicas para reconocer gestos con las manos, las primeras fueron basadas en contacto y le siguieron las basadas en la visi\'on, debido al avance de la tecnolog\'ia estas fueron las m\'as aceptadas. 

Aunque existen diversos m\'etodos para el reconociento de gestos con las manos, con buena precisi\'on, sigue siendo un problema abierto ya que no es f\'acil tener un sistema que se adecue a todo tipo de situaciones como: amigable con el usuario, invariante a la iluminaci\'on, rotaci\'on, al fondo, que funcione en tiempo real o cuando exista oclusi\'on.

\section{Objetivo general}\label{Objetivo general}
 
Desarrollar un sistema que permita controlar la computadora haciendo uso de gestos con las manos, estáticos y dinámicos. El sistema debe ser robusto, funcionar en circunstancias de baja iluminación, cuando exista oclusión en gestos dinámicos.

\section{Objetivos especificos}\label{bjetivos especificos}

\begin{itemize}
	\item Identificar los m\'etodos actuales de reconocimiento de gestos, est\'aticos y din\'amicos cuando existe baja iluminación  y en el caso de los gestos dinámicos cuando existe oclusión. 
	
	\item  Obtener conocimiento acerca del funcionamiento de sistema Microsoft Kinect.
	
	\item Desarrollar un sistema de reconocimiento de gestos estáticos y dinámicos, fusionando la información de los sensores de  profundidad de dos dispositivo kinect. El sistema desarrollado deberá funcionar en circunstancias de baja iluminación y también cuando existe oclusión, causada por los dedos. 
	
	\item Analizar el sistema dise\~nado, en cuanto a su eficiencia presentada en base al reconocimiento de los gestos y tiempo de respuesta, en circunstancias de baja iluminación y oclusión. En el análisis del sistema se usar\'a información real.  
	
	\item Comparar los modelos propuestos  con los existentes, en base al tiempo de respuesta y la eficiencia en cuanto al reconocimiento del gesto. 
\end{itemize}


\section{Limitaciones y suposiciones}\label{Limitaciones y suposiciones}

Gran porcentaje de los trabajos previos en el \'area de reconocimiento de gestos con las manos basados en el modelo de la visi\'on  utilizan c\'amaras digitales o c\'amaras web. Esta investigaci\'on utiliza el dispositivo Kinect, para obtener la informaci\'on de entrada del sistema. 

De  manera que las limitaciones del sistema propuesto est\'an dadas por las características dicho dispositivo, tales como la distancia a la que se encuentra el dispositivo con el usuario (poner la distancia), la resoluci\'on de las im\'agenes a color (poner resolucion) y la resoluci\'on del sensor infrarrojo (poner resolucion).\\
Tambi\'en el sistema depende de dos sensores Kinect, que se utilizar\'an en el caso que exista oclusi\'on. \\
Otra limitante es el número de gestos que podr\'a reconocer el sistema.

Se supone el área de trabajo como un cuarto estándar con buena iluminación (enfocado a pruebas con la cámara color del sistema Kinect)

\section{Reconocimiento de gestos con la manos}\label{Reconocimiento de gestos con las mannos} 


La definición de gestos \citep{Mitra2007} son movimientos del cuerpo expresivos y significativos que involucran a los dedos, manos, brazos, cabeza, cara o cuerpo con la intención de transmitir información relevante o de interactuar con el ambiente. De acuerdo con la literatura \citep{Mitra2007} los gestos con las manos se clasifican en estáticos y dinámicos, los primeros están definidos como la posición y orientación de la mano en el espacio manteniendo esta pose durante cierto tiempo, por ejemplo para hacer una se\~nal de aventón, a diferencia de los gestos dinámicos donde hay movimiento de la pose, un ejemplo  es cuando mueves la mano en se\~nal de adiós. De aquí en adelante entiéndase el término gestos con las manos, como gestos. 

El reconocimiento de gestos se divide en tres fases\cite{Rautaray2012}, detección o segmentación; extracci\'on de caracter\'isticas seguimiento; dependiendo si los gestos son dinámicos, por último la etapa final el reconocimiento del gesto.  
Este se clasifican en dos modelos, basados en la visi\'on y en contacto, esta clasificaci\'on depende de la manera en que son capturados los datos, es decir la forma en que se obtiene el gesto, para posteriormente poderlo reconocer. 

Los primeros acercamientos para llevar acabo el reconocimiento de gestos fue usando modelos de contacto \cite{Rautaray2012} y \cite{Nayakwadi2014}, como su nombre lo dice utilizan dispositivos que est\'an en contacto f\'isico con la mano del usuario, esto para capturar el gesto a reconocer, por ejemplo existen guantes de datos, marcadores de colores, acelerómetros y pantallas multi-touch, aunque estos no son tan aceptados pues entorpecen la naturalidad entre la interacción del humano y la computadora. Los modelos basados en la visi\'on surgieron como respuesta a esta desventaja, estos utilizan cámaras para extraer la información necesaria para realizar el reconocimiento, los dispositivos van desde c\'amaras web hasta algunas más sofisticadas por ejemplo c\'amaras de profundida.  

En este trabajo, se toma el enfoque basado en la visi\'on ya que se quiere obtener un sistema que para el usuario sea facil de interactuar, y esta interacci\'on sea natural y una manera de lograr esto es tomando este enfoque.  estos tienen mayor complejidad (acomodra este parrafo :P)


Los métodos basados en la visión se pueden representar por dos modelos \citep{Rautaray2012}, los basados en 3D, da una descripción espacial en 3D de la mano, y los basados en apariencia, como su nombre lo dice se basan en la apariencia de la mano. Los modelos basados en apariencia se dividen en dos categorías, los estáticos (modelo de silueta, de contorno deformables) y de movimiento (de color y movimiento).

\subsection{Etapas del reconocimiento de gestos}
Enseguida se describen las etapas del reconocimiento de gestos (detección, seguimiento y reconocimiento), con los métodos para llevar cada una de estas.  



\subsubsection{Detección}

En esta etapa se detecta y segmenta la información relevante de la imagen (la mano), con la del fondo, existen distintos métodos para obtener dichas características como la de color de la piel, forma, movimiento, entre otras que generalmente son combinaciones de alguna de estas, para obtener un mejor resultado. Enseguida se describe brevemente cada una de estas.  
\begin{itemize}
\item Color de la piel: Se basa principalmente en escoger un espacio del color, es una organización de colores especifica; como; RGB (rojo, verde, azul), RG (rojo, green), YCrCb (brillo, la diferencia entre el brillo y el rojo, la diferencia entre el brillo y el azul), etc. La desventaja es que si es color de la piel es similar al fondo, la segmentación no es buena, la forma de corregir esta segmentación es suponiendo que el fondo no se mueve con respecto a la cámara.
\item Forma: Extrae el contorno de las imágenes, si se realiza correctamente se obtiene el contorno de la mano. Aunque si se toman las yemas de los dedos como características, estas pueden ser ocluidas por el resto de la mano, una posible solución es usar más de una cámara.  
\item Valor de pixeles: Usar imágenes en tonos de gris para detectar la mano en base a la apariencia y textura, esto se logra entrenando un clasificador con un conjunto de imágenes.
	\item Modelo 3D: Depende de cual modelo se utilice, son las características de la mano requeridas. 
	\item Movimiento: Generalmente esta se usa con otras formas de detección ya que para utilizarse por sí sola hay que asumir que el único objeto con movimiento es la mano.
\end{itemize}

\subsubsection{Seguimiento}

Consiste en localizar la mano en cada cuadro (imagen). Se lleva acabo usando los métodos de detección si estos son lo suficientemente rápidos para detectar la mano cuadro por cuadro. Se explica brevemente los métodos para llevar a cabo el seguimiento. 
\begin{itemize}
	\item Basado en plantillas: Este se divide en dos categorías (Características basadas en su correlación y basadas en contorno), que son similares a los métodos de detección, aunque supone que las imágenes son adquiridas con la frecuencia suficiente para llevar acabo el seguimiento. Características basadas en su correlación, sigue las características a través de cada cuadro, se asume que las características aparecen en mismo vecindario. Basadas en contorno, se basa en contornos deformables, consiste en colocar el contorno cerca de la región de interés e ir deformando este hasta encontrar la mano. 
	\item Estimación óptima: Consiste en usar filtros Kalman, un conjunto de ecuaciones matemáticas que proporciona una forma  computacionalmente eficiente y recursiva de estimar el estado de un proceso, de una manera que minimiza la media de un error cuadrático, el filtro soporta estimaciones del pasado, presente y futuros estados, y puede hacerlo incluso cuando la naturaleza precisa del modelo del sistema es desconocida;  para hacer la detección de características en la trayectoria. 
	\item Filtrado de partículas: Un método de estimación del estado de un sistema que cambia a lo largo del tiempo, este se compone de un conjunto de partículas (muestras) con pesos asignados, las partículas son estados posibles del proceso. Es utilizado cuando no se distingue bien la mano en la imagen. Por medio de partículas localiza la mano la desventaja es que se requieren demasiadas partículas, y el seguimiento se vuelve imposible. 
	\item Camshift: Busca el objetivo, en este caso la mano, encuentra el patrón de distribución mas similar en una secuencia de imágenes, la distribución puede basada en el color. 
\end{itemize}

\subsubsection{Reconocimiento}

Es la clasificación del gesto, la etapa final del reconocimiento, la clasificación se puede hacer dependiendo del gesto. Para gestos estáticos basta con usar algún clasificador o empatar el gesto con una plantilla. En los dinámicos se requiere otro tipo de algoritmos de aprendizaje de máquina. A continuación se encuentran los principales métodos para llevar acabo el reconocimiento del gestos. 
\begin{itemize}
	\item K-medias: Consiste en determinar los $k$ puntos llamados centros para minimizar el error de agrupamiento, que es la suma de las distancias de todo los puntos al centro de cada grupo. El algoritmo empieza localizando aleatoriamente $k$ grupos en el espacio espectral. Cada p\'ixel en la imagen de entrada es entonces asignadas al centro del grupo mas cercano  
	\item {K-vecinos cercanos} (KNN, por sus siglas en ingl\'es): 
	Este es un método para clasificar objetos basado en las muestras de entrenamiento en el espacio de características. 
	\item Desplazamiento de medias: Es un método iterativo que encuentra el máximo en una función de densidad dada una muestra estadística de los datos.
	\item Máquinas de soporte vectorial (SVM, por sus siglas en ingl\'es). Consiste en un mapeo no lineal de los datos de entrada a un espacio de dimensi\'on m\'as grande, donde los datos pueden ser separados de forma lineal.  
	\item Modelo oculto de Markov (HMM, por sus siglas en ingl\'es) es definido como un conjunto de estados donde un estado es el estado inicial, un conjunto de símbolos de salida y un conjunto de estados de transición. En el reconocimiento de gestos se puede caracterizar a    los estados como un conjunto de las posiciones de la mano; las  transiciones de los estados como la probabilidad de transición de cierta posición de la mano a otra; el símbolo de salida como una postura especifica y la secuencia de los símbolos de salida como  el gesto de la mano.   
	\item Redes neuronales con retraso: Son una clase de redes neuronales artificiales que se enfocan en datos continuos, haciendo que el sistema sea adaptable para redes en linea y les da ventajas sobre aplicaciones en tiempo real. 
\end{itemize}  

\section{Estado del arte} 

\section{Organizaci\'on de la tesis}


%This text is normal size, but
%\large
%all of this text is pretty big!
%Except for a couple
%{\Large really} big
%{\LARGE obnoxiously} large
%{\Huge words}.
%You can get back down to size with some
%{\normalsize normal text}.
%If you're feeling
%{\small small}, you can get really
%{\tiny tiny}, but you can also imitate other sizes like
%{\footnotesize footnotes} and
%{\scriptsize subscripts and superscripts}

\newpage
%%=====================================================
